% Exam Template for UMTYMP and Math Department courses
%
% Using Philip Hirschhorn's exam.cls: http://www-math.mit.edu/~psh/#ExamCls
%
% run pdflatex on a finished exam at least three times to do the grading table on front page.
%
%%%%%%%%%%%%%%%%%%%%%%%%%%%%%%%%%%%%%%%%%%%%%%%%%%%%%%%%%%%%%%%%%%%%%%%%%%%%%%%%%%%%%%%%%%%%%%

% These lines can probably stay unchanged, although you can remove the last
% two packages if you're not making pictures with tikz.
\documentclass[11pt]{exam}
\RequirePackage{amssymb, amsfonts, amsmath, latexsym, verbatim, xspace, setspace}
\RequirePackage{tikz, pgflibraryplotmarks}

% By default LaTeX uses large margins.  This doesn't work well on exams; problems
% end up in the "middle" of the page, reducing the amount of space for students
% to work on them.
\usepackage[margin=1in]{geometry}
\usepackage{enumerate}
\usepackage{amsthm}

\theoremstyle{definition}
\newtheorem{soln}{Solution}

% Here's where you edit the Class, Exam, Date, etc.
\newcommand{\class}{Math 407 Section 1}
\newcommand{\term}{Fall 2022}
\newcommand{\examnum}{Exam I}
\newcommand{\examdate}{February 22, 2022}
\newcommand{\timelimit}{75 Minutes}
\newcommand{\ol}[1]{\overline{#1}}

% For an exam, single spacing is most appropriate
\singlespacing
% \onehalfspacing
% \doublespacing

% For an exam, we generally want to turn off paragraph indentation
\parindent 0ex

\begin{document} 

% These commands set up the running header on the top of the exam pages
\pagestyle{head}
\firstpageheader{}{}{}
\runningheader{\class}{\examnum\ - Page \thepage\ of \numpages}{\examdate}
\runningheadrule

\begin{flushright}
\begin{tabular}{p{2.8in} r l}
\textbf{\class} & \textbf{Name (Print):} & \makebox[2in]{\hrulefill}\\
\textbf{\term} &&\\
\textbf{\examnum} & \textbf{Student ID:}&\makebox[2in]{\hrulefill}\\
\textbf{\examdate} &&\\
\textbf{Time Limit: \timelimit} % & Teaching Assistant & \makebox[2in]{\hrulefill}
\end{tabular}\\
\end{flushright}
\rule[1ex]{\textwidth}{.1pt}


This exam contains \numpages\ pages (including this cover page) and
\numquestions\ problems.  Check to see if any pages are missing.  Enter
all requested information on the top of this page, and put your initials
on the top of every page, in case the pages become separated.\\

You may \textit{not} use your books or notes on this exam.  However, you may use a single, handwritten, one-sided notesheet and a \textit{basic} calculator.\\

You are required to show your work on each problem on this exam.  The following rules apply:\\

\begin{minipage}[t]{3.7in}
\vspace{0pt}
\begin{itemize}

%\item \textbf{If you use a ``fundamental theorem'' you must indicate this} and explain
%why the theorem may be applied.

\item \textbf{Organize your work}, in a reasonably neat and coherent way, in
the space provided. Work scattered all over the page without a clear ordering will 
receive very little credit.  

\item \textbf{Mysterious or unsupported answers will not receive full
credit}.  A correct answer, unsupported by calculations, explanation,
or algebraic work will receive no credit; an incorrect answer supported
by substantially correct calculations and explanations might still receive
partial credit.  This especially applies to limit calculations.

\item If you need more space, use the back of the pages; clearly indicate when you have done this.

\item If the problem asks for a proof, be sure to carefully justify your work, including any theorems from class.

Note: you may NOT use a theorem or result from class to prove something when it makes the problem entirely trivial.  If you are unsure whether a particular theorem or result is allowed, just ask!

\end{itemize}

Do not write in the table to the right.
\end{minipage}
\hfill
\begin{minipage}[t]{2.3in}
\vspace{0pt}
%\cellwidth{3em}
\gradetablestretch{2}
\vqword{Problem}
\addpoints % required here by exam.cls, even though questions haven't started yet.	
\gradetable[v]%[pages]  % Use [pages] to have grading table by page instead of question

\end{minipage}
\newpage % End of cover page

%%%%%%%%%%%%%%%%%%%%%%%%%%%%%%%%%%%%%%%%%%%%%%%%%%%%%%%%%%%%%%%%%%%%%%%%%%%%%%%%%%%%%
%
% See http://www-math.mit.edu/~psh/#ExamCls for full documentation, but the questions
% below give an idea of how to write questions [with parts] and have the points
% tracked automatically on the cover page.
%
%
%%%%%%%%%%%%%%%%%%%%%%%%%%%%%%%%%%%%%%%%%%%%%%%%%%%%%%%%%%%%%%%%%%%%%%%%%%%%%%%%%%%%%

\begin{questions}

\addpoints

\question[10]\mbox{}
\textbf{TRUE or FALSE!}  Write  TRUE if the statement is true.  Otherwise, write FALSE.  Your response should be in ALL CAPS.  No justification is required.
\begin{enumerate}[(a)]
\item  The group $\mathbb Q$ with binary operation $+$ is a cyclic group.
\vspace{1.3in}
\item  Every group of order $7$ is cyclic.
\vspace{1.3in}
\item  If $G$ is non-Abelian and $H\leq G$ then $H$ is also non-Abelian
\vspace{1.3in}
\item  Every nontrivial group has a nontrivial proper subgroup.
\vspace{1.3in}
\item  If $G$ is a group and $x\in G$ with $x^2 = x$ then $x$ is the identity
\vspace{1.3in}
\end{enumerate}

\newpage
\question[10]\mbox{}
\begin{enumerate}[(a)]
\item  State Cayley's Theorem
\vspace{2in}
\item  Write the definition of a homomorphism from a group $G$ to a group $H$.
\vspace{2in}
\item  Give an example of a group of order $6$ which is not cyclic.
\end{enumerate}

\newpage
\question[10]\mbox{}
\begin{enumerate}[(a)]
\item  Write down (up to isomorphism) all Abelian groups of order $64$.
\vspace{2.5in}
\item Write down $\mathbb Z_3\oplus \mathbb Z_4\oplus \mathbb Z_6$ in prime divisor form.
\vspace{2.5in}
\item Write down $\mathbb Z_3\oplus \mathbb Z_4\oplus \mathbb Z_6$ in invariant factor form.
\end{enumerate}

\newpage
\question[10]\mbox{}
$G=\{(a,b): \text{$a,b$ integers},\ 0\leq a < 7,\ 0\leq b < 3\}$

with the associative binary operation

$$(a,b)\ast (c,d)  = (a +_7 (2^bc), b +_3 d).$$

For example
$$(3,2)*(5,1) = (3 +_7 (2^25),2+_31) = (3 +_7 20, 0) = (2,0).$$

\begin{enumerate}[(a)]
\item Show that $G$ has an identity element
\vspace{1.5in}
\item Show each element of $G$ has an inverse
\vspace{2in}
\item Show that $G$ is not Abelian
\vspace{2in}
\item Find the order of the element $(1,1)$ in $G$.  Show your work.
\vspace{2in}
\end{enumerate}

\newpage
\question[10]\mbox{}
Let $G$ and $G'$ be groups with identities $e$ and $e'$, respectively.  Also let $\varphi: G\rightarrow G'$ be a group homomorphism.  Prove each of the following statements

\begin{enumerate}[(a)]
\item $\varphi(e) = e'$
\vspace{2in}
\item $\varphi(a^{-1}) = (\varphi(a))^{-1}$
\vspace{2in}
\item the kernel of $\varphi$
$$\ker(\varphi) = \{a\in G: \varphi(a)=e'\}$$
is a subgroup of $G$
\vspace{2in}
\end{enumerate}


\newpage
\question[10]\mbox{}

Suppose that $G$ is a finite group (possibly non-Abelian).  Show that if $\varphi$ is a homomorphism

$$\varphi: G\rightarrow\mathbb Z$$

then $\varphi(x) = 0$ for all $x\in G$.  In other words, the only homomorphism from $G$ to $\mathbb Z$ is the trivial one.


\end{questions}
\end{document}
